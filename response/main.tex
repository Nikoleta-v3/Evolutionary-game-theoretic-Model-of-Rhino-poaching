\documentclass[10pt]{article}

\usepackage[margin=2.5cm, includefoot, footskip=30pt]{geometry}

\title{An Evolutionary Game Theoretic Model of Rhino Horn Devaluation RESPONSE
       TO REVIEWS}
\author{Nikoleta E. Glynatsi, Vincent Knight, Tamsin E. Lee} %TODO Authors

\begin{document}

We open this response by thanking the reviewers for their thoughtful comments
and suggested. In fact, we have fully taken their comments on board and
made what we hope will be considered the major modifications.

The main modification has been to the utility model itself: upon reflection we
have included another component of the poaching process to the utility model. By
considering the time it takes to kill a rhino this ensures a heightened risk to
indiscriminate poachers as they will kill all rhinos they see they will spend
more time in the park at risk of being caught. This has en immediate effect of
modifying the conclusions of our first submission: all strategies can
be evolutionarily stable.

Given this major modification we will now respond to each comment/suggestion
below:

\begin{quote}
    Ln 20 – 40, pg 6, the variables talk about "proportion of horn gained" but I think a more general way to think of the variable is "proportion of value gained from the horn". For example, de-horning clearly leads to proportion of horn, but for other management interventions such as dye it may be a proportion of devaluing rather than quantity. Think it would be useful to say something like this model holds for any management action that proportionally devalues or removes a quantity of horn from the market, we will use the word "horn gained" because we are thinking about "dehorning" but the model is more general, or something to that effect.
\end{quote}

This has been addressed by making the suggested substitution and including the
opening comment but also ensuring
that unless dehorning is specifically intended that the term devaluing is used.

\end{document}
