\documentclass[10pt]{article}

\usepackage[margin=2.5cm, includefoot, footskip=30pt]{geometry}

\title{An Evolutionary Game Theoretic Model of Rhino Horn Devaluation RESPONSE
       TO REVIEWS}
\author{Nikoleta E. Glynatsi, Vincent Knight, Tamsin E. Lee} %TODO Authors

\begin{document}

We open this response by thanking the reviewers for their thoughtful comments
and suggested. In fact, we have fully taken their comments on board and
made what we hope will be considered the major modifications.

The main modification has been to the utility model itself: upon reflection we
have included another component of the poaching process to the utility model. By
considering the time it takes to kill a rhino this ensures a heightened risk to
indiscriminate poachers as they will kill all rhinos they see they will spend
more time in the park at risk of being caught. This has en immediate effect of
modifying the conclusions of our first submission: all strategies can
be evolutionarily stable.

Given this major modification we will now respond to each comment/suggestion
below:


\section{Reviewer 1}

\begin{verbatim}
    Ln 20 – 40, pg 6, the variables talk about "proportion of horn gained" but I
    think a more general way to think of the variable is "proportion of value
    gained from the horn". For example, de-horning clearly leads to proportion
    of horn, but for other management interventions such as dye it may be a
    proportion of devaluing rather than quantity. Think it would be useful to
    say something like this model holds for any management action that
    proportionally devalues or removes a quantity of horn from the market, we
    will use the word "horn gained" because we are thinking about "dehorning"
    but the model is more general, or something to that effect.
\end{verbatim}

This has been addressed by making the suggested substitution and including the
opening comment but also ensuring
that unless dehorning is specifically intended that the term devaluing is used.

\begin{verbatim}
    Eqn 5, There are many mathematically valid choices for how to write this
    equation, including yours, but I think this way of expressing it is less
    clear than some alternatives.  I think it would be clearer to write this
    equation in one of the following ways, (1-r) + (1-x)(1- r + r sigma_r) or 1
    – r + r sigma_r (1-x) to me your choice of expression obfuscates the meaning
    and creates an awkward nested parenthesis. Of the two expressions above, the
    first one starts from the point of view of the proportion of poachers, the
    alternative equation starts from the point of view of the rhino (i.e. whole
    horns are harvested from both types (1-r) and r sigma_r (1-x) is the
    additional value harvested by the indiscriminate poachers. The second is the
    simplest expression in my opinion, but I can see reasons for why you might
    prefer the former one.
\end{verbatim}

Thank you to the reviewer for pointing this out: we have made the suggested
change.

\begin{verbatim}
    >> Ln 55-56 pg 6, can you avoid introducing the variable capital x. Simply
    say as you basically do in the following line, that x uniquely determines
    the proportion of both types and therefore describes the state of the
    population of poachers. The added variable is unnecessary unless you are
    going to use it heavily later.

    >> Similar to the capital X comment this sentence is also confusing. Just
    introduce s.

\end{verbatim}

We have made this suggested change.

\begin{verbatim}

    >> Eqn 6 is confusing as to how it should exactly be thought of and how it
    relates to supply and demand relationships. It is not clear how fig.2 shows
    how "fig 2 verifies that the gain curve corresponds to a demand curve". How
    exactly does it do this?
\end{verbatim}

Eqn 6 attempts to capture that by devaluing raw rhino horn (increasing \(r\))
its scarcity in the market increases which in turn increases its `price'.
We have attempted to clarify this further in the text.

\begin{verbatim}
    >> Eqn 9, this is a bit strange to me, why is this an exponent? Consider
    that each rhino dehorned caused a unit of money that could go to security,
    then dollars to security is directly proportional to (1 – r) so why not a
    constant multiple? Presumably if beta < 1 this could denote a diminishing
    return on police efficacy

    >> Eqn 8, I do not understand this equation. What is this the cost of?
    Searching for rhinos?  Presumably this could be measured in the amount of
    time it took to find a rhino comparatively if you were passing on rhinos
    that have been devalued. If we were assuming poachers encountered rhinos
    randomly via laws of mass action for every time unit it took for
    indiscriminate poachers to find a rhino it would take 1/(1-r) units of time
    for an indiscriminate poacher to find a rhino that hasn't been devalued. Is
    this how equation 8 is being calculated? If so I do not follow the
    derivation. I’m also concerned with double counting. Your benefit function
    includes the added benefit from poaching rhinos indiscriminately by getting
    more rhinos. Please walk the reader through this equation.  Is it search
    cost? Is this the same cost for an individual of both strategies? I’m not
    really sure what you are assuming here.

    >> Eqn 5 – 11, Handling time of killing and processing a rhinos and the
    increased risk of being caught due to killing more rhinos (dehorned rhinos)
    do not appear to be factored into the model. As a result I question whether
    the main result is an artefact of this model set-up. I’m not sure I follow
    how an discriminant poacher gets any benefit from their actions in this
    model (with the exception of H). I don't think supply and demand is the main
    reason managers argue for this strategy, so the effect of H is less
    important to explore than handling time and risk to poachers, in my opinion.
    From a game theoretic sense, unless I am missing something, it is obvious
    that increases in H with r wouldn't affect the stable strategies because all
    individuals are affected by the price of the value of horn equally. So only
    through risk of exposure and added costs through handling time of killing
    less valuable rhinos can one argue for dehorning, which this paper doesn't
    explore.

    >> Eqn 11 confuses me as I thought time would be the unit of cost of
    searching for rhinos, as can be seen from my comment above I thought
    something like these equations would be the costs. Basically I was expecting
    equation 11 to be the costs. It seems like we have similar thoughts. Psi
    actually seems to just be proportional to the number of rhinos being
    poached, but you call it the cost, which is confusing to me. Am I missing
    something here?
\end{verbatim}

We have reflected on these raised points extensively and have explicitly
addressed the points raised. Indeed, as was the model did not account for the
time related to killing and processing rhinos which in the case of the
indiscriminate poachers would correlate with an augmented risk.

This has now been modelled using a stochastic walk, this has been
explained in the text and a diagram has also been included.

The reviewer was completely correct in their insight about the conclusions and
a new theoretic result has been obtained that describes the conditions for which
devaluation will lead to selective poachers.

We thank the reviewer for very helpfully pointing this out and feel that this
has greatly improved the paper including the conclusions that detail the factors
that influence selective behaviour.

\begin{verbatim}
    >> When introducing H I think you need a reference for its choice. Below are
    some modelling papers where supply and demand affecting poaching

    - [Economics of Antipoaching Enforcement and the Ivory Trade
    Ban](https://www.jstor.org/stable/1244594?seq=1#page_scan_tab_contents) -
    [Protecting the African elephant: A dynamic bioeconomic model of ivory
    trade](http://www.unece.lsu.edu/responsible_trade/documents/2008/rt08_33.pdf)


    >> There are many papers modelling poaching in the rhino trade system. You
    might want to consider going through them a bit more systematically and
    describing how your work sits within the broader rhino horn modelling
    literature (possibly in the discussion).  Here are some examples (but there
    are many more) to seed your search

    - [Debunking the myth that a legal trade will solve the rhino horn crisis: A
    system dynamics model for market
    demand](http://www.saeon.ac.za/enewsletter/archives/2015/october2015/images/0300.pdf)
    - [Identification of policies for a sustainable legal trade in rhinoceros
    horn based on population projection and socioeconomic
    models.](https://www.ncbi.nlm.nih.gov/pubmed/25331485) - [Trading on
    extinction: An open-access deterrence model for the South African abalone
    fishery](http://www.scielo.org.za/scielo.php?script=sci_arttext&pid=S0038-23532016000200019)
    - [How many to dehorn? A model for decision-making by rhino
    managers](https://www.cambridge.org/core/journals/animal-conservation-forum/article/how-many-to-dehorn-a-model-for-decisionmaking-by-rhino-managers/63CBEF4F8929487E3420AD41BADF601B)


    >> It would be good to cite the anthropogenic Allee effect in either line 42
    of the intro or in the discussion as this is one of the main hypothesised
    drivers in the mathematical modelling literature on what causes rhino horn
    to be so valuable.  The theory is mathematically described in

    - [High prices for rare species can drive large populations extinct: the
    anthropogenic Allee effect
    revisited.](https://www.ncbi.nlm.nih.gov/pubmed/28669883)

    >> and first biologically hypothesized here

    - [Rarity Value and Species Extinction: The Anthropogenic Allee
    Effect](http://journals.plos.org/plosbiology/article?id=10.1371/journal.pbio.0040415)

    >> Ln 20 – 43 pg 3. The issue of opportunistic exploitation is highly
    relevant to the game you present. In fact I think it may be the logic behind
    a lot of what's driving your results. It's not exactly the same because the
    authors think of opportunistic exploitation as a multi-species problem, but
    it is very similar because you could think of dehorned and horned rhino as
    two species. The basic idea is that while hunting for one type you will kill
    the other if it's there and easy to kill without much added cost.  Ln 20 -43
    seems like a natural place to talk about this, but it could be talked about
    later in the discussion

    - [Opportunistic exploitation: an overlooked pathway to
    extinction](https://www.sciencedirect.com/science/article/pii/S0169534713000712)
\end{verbatim}

Thank you for these suggestions we have included the above citations.

\section{Reviewer 2}

\begin{verbatim}
    >> I think the most problematic assumption is that rhino population dynamics
    is assumed to be stable through time. Obviously, this won't be the case and
    that could change everything because the increasing rarity of rhino could
    increase the price of horn, devalued or not, and therefore the relationship
    between costs and benefits that would be challenging to predict. I think
    addressing this question could definitely increase dramatically the novelty
    of this study, which will be beneficial from an ecological and a modelling
    point of views.
\end{verbatim}

Whilst the rhino population dynamics are not modelled explicitly we agree this
would be an interesting avenue to consider, we have included this in the
discussion. However, we feel that for the purpose of analysis considered at this
stage assuming the dynamics of the rhino population and thus the cost/benefit to
the poachers is sufficiently captured by the supply/demand model: whilst not
considered specifically the rarity of rhino is consider by the \((1-r)\) term
and the behaviour of the poachers. This can also be seen in some of the new
scenarios that have been included.

\begin{verbatim}
    >> The authors have also identified some situations were selective strategy
    could be more beneficial than indiscriminate. I think it could be
    interesting to look at the literature in which kind of situations we really
    are, and if these conditions found theoretically can be really encountered
    on the field.
\end{verbatim}

We have not addressed this as a lot of these details are purposely kept
confidential ... % TODO Tamsin could you perhaps word a response here please?


\begin{verbatim}
    >> Regarding the length of the paper, I am not 100% sure that the analysis
    for situations where everyone has selective or indiscriminate strategy
    really brings something.  This highlight that this study is maybe not deep
    enough, because classically, these analyses would be moved in supplementary
    materials. I understand the point of showing the difference with mixed
    strategies, which won't be stable, but I still think these analyses on pure
    strategies are not very insightful.
\end{verbatim}

Thank you for this comment which encouraged us to reflect on the length of the
paper, as a result we have modified the model to only consider a disincentive
and furthermore only a single theoretic result is presented. This specifically
offers insight regarding the conditions required for selective poachers to
subsist.

\begin{verbatim}
    >> Finally, from a presentation point of view, I think that figure 1 is not
    needed
\end{verbatim}

We have removed this figure.

\begin{verbatim}
    >> Figure 2 and 3 could be merged.

    >> Legends and axes names need to be improved to make the figures readable
    without going back to the text.
\end{verbatim}

Figure 3 has been replaced with another Figure relevant to the new model and the
axes and legends are now hopefully clear.

\begin{verbatim}
    >> I think that a summary figure of the selected strategies
    (without discentive) could be also an interesting addition.
\end{verbatim}

We have included a figure that describes the strategies as requested.
\end{document}
