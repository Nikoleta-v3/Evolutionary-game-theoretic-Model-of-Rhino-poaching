\documentclass[10pt]{article}

\usepackage[margin=2.5cm, includefoot, footskip=30pt]{geometry}

\title{An Evolutionary Game Theoretic Model of Rhino Horn Devaluation RESPONSE
       TO REVIEWS}
\author{Nikoleta E. Glynatsi, Vincent Knight, Tamsin E. Lee} %TODO Authors

\begin{document}

We open this response by thanking the reviewers for their thoughtful comments
and suggested. In fact, we have fully taken their comments on board and
made what we hope will be considered the major modifications.

The main modification has been to the utility model itself: upon reflection we
have included another component of the poaching process to the utility model. By
considering the time it takes to kill a rhino this ensures a heightened risk to
indiscriminate poachers as they will kill all rhinos they see they will spend
more time in the park at risk of being caught. This has en immediate effect of
modifying the conclusions of our first submission: all strategies can
be evolutionarily stable.

Given this major modification we will now respond to each comment/suggestion
below:


\section{Reviewer 1}

\begin{verbatim}
    Ln 20 – 40, pg 6, the variables talk about "proportion of horn gained" but I
    think a more general way to think of the variable is "proportion of value
    gained from the horn". For example, de-horning clearly leads to proportion
    of horn, but for other management interventions such as dye it may be a
    proportion of devaluing rather than quantity. Think it would be useful to
    say something like this model holds for any management action that
    proportionally devalues or removes a quantity of horn from the market, we
    will use the word "horn gained" because we are thinking about "dehorning"
    but the model is more general, or something to that effect.
\end{verbatim}

This has been addressed by making the suggested substitution and including the
opening comment but also ensuring
that unless dehorning is specifically intended that the term devaluing is used.

\begin{verbatim}
    Eqn 5, There are many mathematically valid choices for how to write this
    equation, including yours, but I think this way of expressing it is less
    clear than some alternatives.  I think it would be clearer to write this
    equation in one of the following ways, (1-r) + (1-x)(1- r + r sigma_r) or 1
    – r + r sigma_r (1-x) to me your choice of expression obfuscates the meaning
    and creates an awkward nested parenthesis. Of the two expressions above, the
    first one starts from the point of view of the proportion of poachers, the
    alternative equation starts from the point of view of the rhino (i.e. whole
    horns are harvested from both types (1-r) and r sigma_r (1-x) is the
    additional value harvested by the indiscriminate poachers. The second is the
    simplest expression in my opinion, but I can see reasons for why you might
    prefer the former one.
\end{verbatim}

Thank you to the reviewer for pointing this out: we have made the suggested
change.

\begin{verbatim}
    >> Ln 55-56 pg 6, can you avoid introducing the variable capital x. Simply
    say as you basically do in the following line, that x uniquely determines
    the proportion of both types and therefore describes the state of the
    population of poachers. The added variable is unnecessary unless you are
    going to use it heavily later.

    >> Similar to the capital X comment this sentence is also confusing. Just
    introduce s.

\end{verbatim}

We have made this suggested change.

\begin{verbatim}

>> Eqn 6 is confusing as to how it should exactly be thought of and how it
relates to supply and demand relationships. It is not clear how fig.2 shows how
"fig 2 verifies that the gain curve corresponds to a demand curve". How exactly
does it do this?
\end{verbatim}

Eqn 6 attempts to capture that by devaluing raw rhino horn (increasing \(r\))
its scarcity in the market increases which in turn increases its `price'.
We have attempted to clarify this further in the text.

\end{document}
