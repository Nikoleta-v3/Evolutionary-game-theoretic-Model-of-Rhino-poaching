\documentclass{article}
\usepackage[margin=1.5cm, footskip=0pt]{geometry}

\pagestyle{plain}
\setlength{\parindent}{0em}
\setlength{\parskip}{1em}
\renewcommand{\baselinestretch}{1}

\title{An evolutionary game theoretic model of rhino horn devaluation.}
\date{}
\author{Nikoleta E. Glynatsi}


\begin{document}
\maketitle

In recent times, rhino populations are at a critical level due to the demand for rhino
horn and the subsequent poaching. Rhinos now persist in protected areas or on private
land.

Wildlife managers, in charge of these areas, attempt to secure rhinos using approaches
to devalue the horn. However the efficacy of the approach is dependent on the behaviour of the poachers:

\begin{itemize}
    \item They can be`selective poachers’: they will not hunt dehorned rhinos or
    \item `indiscriminate poachers’: they will kill any rhino as it is not
        possibly to fully devalue a horn.
\end{itemize}

Game theory has been used to examine the interaction of poachers and wildlife
managers.
% TODO Cite Tamsin's paper and others perhaps? (Perhaps space is an issue?)
This work is however the first to consider an evolutionary game theoretic model to determine which
strategy is preferred by a poacher in a dynamic manner.

The purpose of the work is to discover whether conditions which encourage the poachers
to behave selectively exist, that is, they only kill those rhinos with full horns.
The work contributes to the discussion of does devaluation work?

This paper shows that devaluing can indeed work, when implemented along with a strong
disincentive framework, such as educational interventions and/or engaging the rural communities
that live with wildlife.

This work contributes to an ongoing discussion surrounding the protection of
endangered animals. Whilst it does not pretend to be the full solution it is an
example of applying mathematics in a way to improve the world around us.
% TODO Nik to check if she's happy with above.

\end{document}
