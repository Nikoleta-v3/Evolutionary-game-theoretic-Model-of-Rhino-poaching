\documentclass{article}
\usepackage[margin=1.5cm, footskip=0pt]{geometry}

\pagestyle{plain}
\setlength{\parindent}{0em}
\setlength{\parskip}{1em}
\renewcommand{\baselinestretch}{1}

\title{An evolutionary game theoretic model of rhino horn devaluation.}
\date{}
\author{Nikoleta E. Glynatsi}


\begin{document}
\maketitle

In recent times, rhino populations are at a critical level due to the demand for rhino
horn and the subsequent poaching. Rhinos now persist in protected areas or on private
land, and require intensive protections.

Wildlife managers, in charge of these areas, attempt to secure rhinos using approaches
to devalue the horn. However the efficacy of the approach is dependent on the behaviour of the poachers:

\begin{itemize}
    \item They can be`selective poachers’: will not hunt dehorned rhinos or
    \item `indiscriminate poachers’: will kill any rhino (even devalued horns still have a small value).
\end{itemize}

Tough game theory has been used to examine the interaction of poachers and wildlife managers,
our work is the first to consider an evolutionary game theoretic model to determine which
strategy is preferred by a poacher in various different populations of poachers. 

The purpose of the work is to discover whether conditions which encourage the poachers
to behave selectively exist, that is, they only kill those rhinos with full horns.
In essence, contribute to the discussion of does devaluation work?

Our paper shows that dehorning can indeed work, when implemented along with a strong
disincentive framework, such as educational interventions or engaging the rural communities
that live with wildlife. \textit{need closing sentence}.


\end{document}