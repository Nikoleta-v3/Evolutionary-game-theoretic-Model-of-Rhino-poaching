\documentclass[10pt]{article}
\title{Cover Letter}
\date{}


\begin{document}
\maketitle

Rhino species populations are at a critical level due to the demand of rhino horn
and the subsequent poaching. Wild life managers, who are in charge of the protected
areas, attempt to secure the life of the animal using several devaluing approaches.
The most common approach is dehorning. Nonetheless, dehorning does not always repel
the poachers. In order for dehorning to be successful a poacher must choose to only
kill rhinos with full horns, thus be a ‘selective poacher'. Indiscriminate poachers
also exist, which are poachers who choose to kill all rhinos they encounter.

In 2016, the interaction between a rhino manager and poachers was investigated for
the first time using a game theoretic model. The interaction was formulated as a
two players normal form game where each player had a set of two actions. A manager
could choose between dehorning or not, and a poacher could choose to behave selectively
or indiscriminately. The paper discussed two of the nash equilibriam of the game
and concluded that poachers will always choose to behave indiscriminately.

In this manuscript, we explore what circumstances would ensure that rhino dehorning
is successful. Using evolutionary game theory we predict the strategies adopted
by new poachers joining an existing population such that poacher outcome is
maximised. This work contributes to ecological management through the understanding
of poacher behaviour on a population dynamic level.

The game considered is not of a two player game
anymore (manager and poacher) but now an infinite population of poachers is considered.
The purpose of the work is to determine which strategy is preferred by a poacher
in various different populations of poachers and whether conditions which encourage
poachers to behave selectively exist. This is explored using both analytical and
numerical methods. Analytical methods include investigating the underlying differential
equations of the population.

The findings confirm that behaving selectively will not persist, not even in a
mixed population using the fact that the only evolutionary stable strategy is the
indiscriminate one. Meaning, that for any given starting population, the poachers
would evolve to adopt an indiscriminate behaviour. However, our results reveal
that it is possible for a population of selective poachers to exist, but for this
to occur a disincentive must be applied to the utility of indiscriminate poachers.
The disincentive factor can have several interpretations; such as engaging the
rural communities that live with wildlife.
\end{document}