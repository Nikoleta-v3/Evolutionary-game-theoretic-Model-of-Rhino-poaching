\documentclass[10pt]{article}
\title{Cover Letter}
\date{}


\begin{document}
\maketitle

Rhino species populations are at a critical level due to the demand of rhino horn
and the subsequent poaching. Wild life managers, who are in charge of the protected
areas, attempt to secure the life of the animal using several devaluing approaches.
The most common approach is dehorning. Nonetheless, dehorning does not always repel
the poachers. In order for dehorning to be successful a poacher must choose to only
kill rhinos with full horns, thus be a `selective poacher'. Indiscriminate poachers
also exist, which are poachers who choose to kill all rhinos they encounter.

In 2016, the interaction between a rhino manager and poachers was investigated for
the first time using a game theoretic model~\cite{Lee}. 
In this manuscript, we explore the population dynamic effects associated to these
interactions. More specifically, the interaction between poachers is investigated
using evolutionary game theory. The game considered is not of a two player game
anymore (manager and poacher) but now an infinite population of poachers is considered. 
The purpose of the work is to determine what circumstances would ensure that rhino
dehorning is successful. This is explored using both analytical and numerical methods.

The findings confirm that behaving selectively will not persist, not even in a
mixed population, using the fact that the only evolutionary stable strategy is the
indiscriminate one. Meaning, that for any given starting population, the poachers
would evolve to adopt an indiscriminate behaviour. However, our results reveal
that it is possible for a population of selective poachers to exist, but for this
to occur a disincentive must be applied to the utility of indiscriminate poachers.
The disincentive factor can have several interpretations; such as engaging the
rural communities that live with wildlife.

The insights gained from this manuscript contribute to ecological management through
the understanding of poacher behaviour on a population dynamic level.

\bibliographystyle{plain}
\bibliography{cover_bibliography.bib}
\end{document}